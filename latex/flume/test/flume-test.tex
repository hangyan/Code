\documentclass{article}
\usepackage{fontspec, xunicode, xltxtra}
\usepackage{listings}
\setmainfont{Hiragino Sans GB} 
\title{flume 测试环境使用文档}
\author{Hang Yan}

\begin{document}
\maketitle
\newpage
\section{简介}
flume是结构化日志的传输通道,集成结构化日志客户端的组件,将日志发送给本机部署的flume后,本机的flume再发往hadoop集群内部的汇聚层flume,最终写入集群
并发送给solr建立索引(云知道)。搭建测试环境时,本机的flume收到结构化日志数据后,不再发往远端,而是直接写入本地磁盘模拟线上存储过程. \\
由于flume传输的数据均为序列化后的二进制数据,所以写到本地的日志文件时不可读的,需要用avro工具将其转化为json数据查看。
\section{依赖}
\subsection{jdk}
flume是java程序(windows任务管理器的程序名即为java。测试机上需要有完整的jdk环境.
\subsection{chmod}
测试flume模拟了线上写入hadoop集群的操作。中间对文件的操作需要调用linux上的chmod命令。linux上自带此命令,windows上如果有cygwin或者git环境的话其所带的bin
目录下也会有此程序,将其加入环境变量中的PATH即可。

\section{启动}
\subsection{windows}
如果jdk版本为1.6.31的话,将整个jdk1.6.0\_31/目录拷贝至apache-flume-1.5.0-SNAPSHOT-bin目录下.则直接点击run-windows.bat即可。
如果不是以上情况,并确定java命令在系统PATH里面,则将bin/flume-ng.bat中的"jdk1.6.0\_31/bin/java.exe" 改为java.exe即可.然互点击run-windows.bat启动.

\subsection{linux}
确保java在PATH中即可,运行start.sh 即可.

\section{运行}
\subsection{文件}
flume启动后,会在本地创建一个data目录用来缓存日志文件,logs/flume.log是正在写的flume日志.如果收到日志后并成功写到本地磁盘将会建立flume文件夹.
如果data和flume占用空间过大,可删除这两个目录。logs下的日志文件过多的话也可删除.每次启动flume时也可删除data和flume文件夹以防止历史数据干扰。
\subsection{页面}
flume启动后,会启动两个端口,一个是用来接收客户端数据的4545端口。另一个是网页端口,用来展示运行时的统计信息.端口号为41414.
此页面展示了一个json文档,推荐浏览器中安装jsonview插件查看.客户端成功连接flume后,此页面会有loglib发送的相关数据的统计.

\section{日志查看}
flume收到的日志文件最终都写在flume文件夹下,正常的目录结构是 flume/s.n/s.v/date/hour/hostname/stable/logavro.timestamp.
s.n s.v是结构化日志的结构名称和版本号。当前是msp/1.date/hour是loglib flush接口所传入的时间戳解析出来的.hostname为flume所获取的本机主机名,
最深层的目录包含current和stable.current是flume正在写的日志文件,所有文件名有tmp后缀,不可查看.一段时间之后会将current里面的文件移到stable里面,
并去除'tmp'后缀.

avro-tool jar包可以用来查看stable里面的的日志内容.具体命令为 \\
\begin{lstlisting}[language=BASH]
java -jar avro-tools.jar tojson [logavro-file]
\end{lstlisting}
可将输出重定向到文本文件中查看。 

\section{停止}
\subsection{windows}
flume 的java 启动窗口,Ctrl-C y 确认即可.
\subsection{linux}
ps aux|grep flume 找到相关的进程号.kill -15 或kill 即可.




\end{document}