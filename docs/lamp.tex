\documentclass[11pt]{article}
\usepackage{CJK}
\usepackage{listings}
\begin{CJK}{UTF8}{gkai}
%Gummi|063|=)
\title{\textbf{Ubuntu10.04下LAMP相关配置}}
\author{闫航}
\date{July 28,2012}
\begin{document}
\maketitle


\section{安装}
ubuntu官方源里有想应软件包提供下载,图形界面下用 Synaptic Package Manager选择相应包下载即可,
或者用命令方式也可。主要方式有两种:\\

\begin{itemize}
\item 分别安装mysql,apche,php
\item 整体安装LAMP。Synaptic里提供了按task安装软件包的方式,选中LAMP Server安装即可。命令方式为\textbf{tasksel install lamp-server}
\end{itemize}

\section{配置}
\begin{enumerate}

\item \textbf{Apache}
\begin{enumerate}
\item PHP整合 \\
默认情况下PHP模块并没有启用。Ubuntu下有两个命令可以用来动态地启用和关闭模块:\textbf{a2enmod}和
\textbf{a2dismod}。a2enmod会显示所有可以启用的模块,输入相应的模块名字即可,重启apache后生效。安装了libapache2-mod-php5之后即可用此命令启用PHP模块。
\end{enumerate}

\item \textbf{Mysql}
\begin{enumerate}
\item 中文支持 \\
需将编码设置为UTF8.可从三个层面来进行设置:\\
一是服务器层面。需要修改my.cnf配置文件:\\
\begin{lstlisting}
 [mysqld] 
 character-set-server=utf8
 collation-server=utf8_general_ci 
 
 [client]
 character-set-server=utf8
\end{lstlisting}
 
重启mysql,然后可以在mysql里面查看相应的编码编码是否设置正确:\\
\begin{lstlisting}
mysql> show variables like 'character_set_%';
+--------------------------+----------------------------+
| Variable_name            | Value                      |
+--------------------------+----------------------------+
| character_set_client     | utf8                       |
| character_set_connection | utf8                       |
| character_set_database   | utf8                       |
| character_set_filesystem | binary                     |
| character_set_results    | utf8                       |
| character_set_server     | utf8                       |
| character_set_system     | utf8                       |
| character_sets_dir       | /usr/share/mysql/charsets/ |
+--------------------------+----------------------------+
\end{lstlisting}
二是在创建数据库的时候指定编码。例如:
\begin{lstlisting}
CREATE DATABASE `test` DEFAULT CHARACTER SET utf8 
COLLATE utf8_general_ci;
\end{lstlisting}
三是对于已经创建好的数据库可以通过SQL语句修改其编码设定:\\
\begin{lstlisting}
ALTER DATABASE db CHARACTER SET utf8 COLLATE utf8_general_ci
ALTER TABLE table CONVERT TO CHARACTER SET utf8 
COLLATE utf8_general_ci
\end{lstlisting}
注意修改了数据库的编码后,原来数据库里面的表的编码并没有改变。也需手动修改。
\end{enumerate}

\item{\textbf{phpmyadmin}} \\
phpmyadmin是一个基于web的mysql管理工具(相应的postgresql的web管理工具为phppgadmin)。
二者是基于同样的原理。ubuntu官方源里已提供相应的包,直接下载安装即可。也可通过源码安装。源码安
装只需将相应文件拷至apache网站目录下即可。而通过官方源安装之后相应的网页文件放置在
/usr/share/phpmyadmin目录下,然后设置apache的网站目录即可通过URL:
\textit{127.0.0.1/phpmyadmin}
进行访问。
\end{enumerate}

\end{CJK}
\end{document}
